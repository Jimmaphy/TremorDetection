\section{Conclusie}
\label{section:conclusion}

Het onderzoek is afgerond,
tijd om terug te blikken naar de vragen die vooraf zijn gesteld.
Alles begon met het idee om geluidsonderdrukking toe te passen op tremoren,
waarna dit hele onderzoek is gestart.
Er is veel geleerd: over tremoren, over EMG's en over prototyping.
Vooraf aan dit alles is de volgende hoofdvraag gesteld:
Is het mogelijk om op technische wijze een Essentiële Tremor te detecteren met de mogelijkheid hierop te reageren?
Hierbij zijn de volgende deelvragen gesteld:

\begin{itemize}
    \item Wat is een Essentiële Tremor?
    \item Hoe worden tremoren gedetecteerd?
    \item Hoe kan de data die uit een detectie komt worden gebruikt?
\end{itemize}

ET is een bewegingsstoornis waarbij ledenmaten onwillekeurig trillen,
zonder verdere symptomen of aandoeningen. 
Belangrijke maatstaven hierbij zijn de frequentie en de kracht van de trilling.
De oorzaak is nog vrijwel onbekend en behandeling zijn gericht op het verminderen van de effecten.

Zowel spier- als hersenactiviteit kan worden gebruikt om ET te meten.
Spieractiviteit is de meest toegankelijke vorm, waarbij het gebruik van een EMG het meest gebruikelijk is.
Deze technologie zet stroompjes in de spieren om in geluid, wat gemeten kan worden.
Er zijn vele implementaties, van grote stationaire systemen tot horloges.

Gezien de EMG geluid gebruikt om tot zijn uiteindelijke waardes te komen,
kan er heel veel data worden verzameld. Ieder meetpunt in de data die wordt gegeneerd kan worden opgeslagen.
Door correct te labelen kan dit een hele waardevolle dataset worden.
Deze set kan worden om eventuele andere tremoren te voorspellen.
En gezien het gaat over geluid, is misschien het idee achter geluidsonderdrukking bij tremoren zo gek nog niet.

Al met al is er te concluderen dat het zeker mogelijk is om op technische wijze ET te detecteren.
Er is veel data die verzameld kan worden op deze manier, dus het ook zeker mogelijk hierop te reageren.
Echter voor toepassingsmogelijkheden is er meer onderzoek nodig.