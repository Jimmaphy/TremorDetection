\section{Toepassing}
\label{section:application}

Tijdens het schrijven van het laatste hoofdstuk van het onderzoek was het doel om de laatste vraag te beantwoorden:
Hoe kan de data die uit een detectie komt worden gebruikt?
Daarbij wordt er gekeken naar hoe de verkregen kennis in de toekomst kan worden gebruikt om te reageren op metingen.

\subsection{Wat is er geleerd?}

In hoofdstuk~\ref{section:prototyping} is er een prototype gemaakt van een EMG.
Dit is gedaan met vrij simpele en goedkope onderdelen,
waar weinig werk nodig was om gegevens te krijgen uit de elektroden.
Dit laat zien dat het detecteren van tremor goed te doen is.
Het voordeel hierbij is dat frequentie en kracht ideaal zijn om te meten \cite{elsevier2022}.

Als dit mogelijk is met een kleine investering van tijd en geld,
dan zou er veel mogelijk moeten zijn als er in de toekomst gekeken wordt naar mogelijkheden met de GENEActiv \cite{sips2024, activinsights2022}.
Door gegevens te registreren, is het mogelijk om een beeld te krijgen van een tremor in verschillende situaties.
Dit helpt om te bepalen hoe ernstig de tremor is, wanneer een patiënt er het meest last van heeft en hoe de tremor verloopt.

\subsection{Kan dit worden gebruikt?}

In de context van het originele idee,
het toepassen van het concept achter geluidsonderdrukking op tremoren, zou dit bruikbaar moeten zijn.
Geluidsonderdrukking neemt opgenomen geluid, draait het om, en speelt af.
Op deze manier valt het geluid weg \cite{bose2023}.
Geleerd hebbende dat een EMG elektrische spanning in spieren omzet naar geluid,
wetende dat een EMG zowel elektrische spanning kan lezen als geven \cite{knf2022,neuro2009,gohel2020,neurostyle2021},
zou dit moeten kunnen werken. Deze statement is een speculatie en heeft meer onderzoek nodig.

Voor Data Science is het zeker mogelijk om deze data te gebruiken.
In een informeel gesprek met Mischa Beckers, hoofd van het lectoraat Data Science aan de HZ University of Applied Sciences,
is geconcludeerd dat er heel veel informatie te verzamelen is, enkel met een EMG.
Door situaties en opdrachten in een grote dataset correct te labelen, kunnen er vele modellen worden gemaakt.