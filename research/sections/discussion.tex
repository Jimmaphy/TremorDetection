\section{Discussie}
\label{section:discussion}

Tijd om informeel terug te kijken op het onderzoek. 
Als mij een aantal maanden geleden was gevraagd hoe het onderzoek zou lopen,
had ik absoluut niet geantwoord dat het op deze manier was.
Ieder stap die ik heb genomen werd met vol enthousiasme ontvangen,
door alle partijen die hebben bijgedragen aan dit document.
Zelf partijen die uiteindelijk niet zijn teruggekomen in het document.
Er is een heel interview geweest met ActivInsights over de GENEActiv;
ondanks dat dit een verkooppraatje was, hebben ze wel de tijd genomen om alles uit te leggen.
Voor een onderzoek geschreven voor 2,5 studiepunt (70 uur),
was het simpelweg te duur en uitgebreid om verder nog iets met hun product te doen.

\subsection*{Persoonlijke Ervaring}

Het doen van een onderzoek op deze schaal was extreem leerzaam,
het is een simplistische manier om te zeggen, maar het is de waarheid.
En dan is dit relatief nog een klein onderzoek; een hobby onderzoek,
het nastreven van een interesse die ontstond door mijn eigen diagnose en een idee dat 's nachts ontstond.

Iedereen is zo behulpzaam geweest tijdens het proces,
en dit zal zeker te maken hebben met het feit dat ik nog op school zit.
Het is mij al eerder duidelijk geworden dat deuren dat meer voor je open staan dan zodra je in het beroepenveld zit.
Daarom ben ik blij dat ik deze kans heb genomen, ondanks dat er weinig nieuwe informatie naar buiten is gekomen.
Uiteindelijk is het een samenvattend onderzoek, op heel hoog niveau.
Maar misschien blijkt het idee dat ik die ene nacht had, zo gek nog niet te wezen.

Tijd is een concept dat altijd te kort komt. Iedereen heeft altijd meer tijd nodig,
en dit onderzoek is daar geen uitzondering in. 70 uur had ik voor dit onderzoek; 2,5 studiepunt.
En ik durf te zeggen dat ik daar flink overheen ben gegaan. Ik heb niet bijgehouden of zo,
en vind het onnodig om met zulke cijfers te gaan pronken. Maar ik denk wel dat het de passie voor het onderwerp liet zien.
En ik denk oprecht, dat ik heel trots mag zijn op het eindresultaat,
komende vanuit een tweedejaars hbo-ICT student.

\subsection*{Snelheid}

Anne Bouwmeester, een medisch student uit mijn vriendenkring, 
heeft voor het onderzoek met mij gespard over het idee.
Ze gaf aan het idee, hoewel niet direct toepasbaar, zeker interessant vond om te onderzoek.
Haar enige zorg was de snelheid. Je moet heel snel kunnen reageren op een trilling.

Uit het onderzoek is gebleken dat trilling van ET tussen de 4Hz en 12Hz \cite{frontiers2022}.
Dit houdt in dat er 4 tot 12 keer per seconde een trilling is.
De snelheid van de seriële poort van de Arduino is in het prototype ingesteld op een snelheid 9600.
Dit is een zogenoemde baudrate; het aantal bits dat per seconde kan worden verstuurd.

Een integer op de Arduino Uno R3 is 2 bytes groot, een float 4 bytes.
Laten we even van de laatste uitgaan,
dan is het theoretische limit $9600 / (4 * 8) = 9600 / 32 = 300$ berichten per seconde.
Het idee om 12 berichten te lezen, en 12 te schrijven; zou prima moeten lukken.

\subsection*{Data Science}

Mischa Beckers, lector Data Science aan de HZ University of Applied Sciences, 
heeft tijdens een gesprek dat wij hadden over de toepasbaarheid van dit project,
een blik geworpen op de data die wordt geproduceerd door een EMG.

Tijdens het gesprek heb ik verteld dat de ontwikkeling van, klachten door, en verloop van ET,
anders is voor vrijwel iedere patiënt. Dit wordt ook ondersteund door onderzoek
\cite{knf2022,erasmus2022,elsevier2022,frontiers2022,sips2024,activinsights2022}.
Dit maakt het moeilijk om voorspellingsmodellen te maken.

Echter zorgt de data die je uit een EMG krijgt, wel voor heel veel testdata.
Dus er is zeker potentie, vooral als de data goed gelabeld is.
Je zou per activiteit een model kunnen maken voor het voorspellen van een tremor.
Het zou zeker interessant zijn om mogelijkheden hierin te bekijken,
net zoals het live kunnen voorspellen van deze trillingen,
dan zou je misschien op nieuwe variabelen kunnen komen die effect hebben op de tremor.
Dit in combinatie met het feit dat alcohol effect heeft op de intensiteit van een tremor \cite{erasmus2022,elsevier2022},
Iets dat mijn neuroloog tijdens mijn eigen diagnose ook aangaf.
Je kan dus echt hele uitgebreide modellen maken en dit bijna eindeloos uitbreiden.

\subsection*{Ethiek}

Een discussie die gevoerd dient te worden,
mocht het originele idee over geluidsonderdrukking verder worden uitgewerkt, is ethiek.
Er moet een manier gevonden om de patiënt een schok de spieren van de patiënt in te sturen,
op de regelmaat van de trillen. Hoe klein deze schok ook wezen mag,
is het de vraag of dit wenselijk en menselijk is.
Ik ben een ICT-student die een uitstapje maakt naar de medische wereld met dit onderzoek,
en heb simpelweg niet de vaardigheid om hier een oordeel op te geven.
Dit zou vanuit iemand anders moeten komen.