\section{Introductie}

Het schooljaar 2023-2024 was pas net begonnen,
het tweede jaar van de opleiding ICT aan de HZ University of Applied Sciences was van start gegaan.
Direct aan het begin van dit jaar is er een vriendschap ontstaan met Jeroen Stout, iemand met Essentiële Tremor.
De impact van deze aandoening was heel duidelijk; veel acties die normaal waren voor anderen, kostte hem veel moeite.
Hij wees mij op het feit dat ik zelf ook symptomen vertoonde van de aandoening.
Een paar maanden later is dit ook daadwerkelijk vastgesteld.

In de maanden tussen het ontstaan van de vriendschap en de diagnose, ontstond er een idee.
Een tremor zijn trillingen in het lichaam, vergelijkbaar met geluid.
In veel koptelefoons zit tegenwoordig geluidsonderdrukking,
een techniek waarbij opgenomen omgevingsgeluid omgedraaid wordt afgespeeld waardoor dit wegvalt\cite{bose2023}.
Zou het mogelijk zijn om dit toe te passen op een Essentiële Tremor?

Gezien mijn beperkte kennis over dit gebied, en de beschikbare tijd voor het onderzoek,
zou het oneerlijk zijn om deze grote vraag in één keer te beantwoorden.
Daarom is er gekozen om de vraag meer te stellen vanuit mijn opleiding, meer vanuit Data Science.
De hoofdvraag voor het onderzoek is:
Is het mogelijk om op technische wijze een Essentiële Tremor te detecteren met de mogelijkheid hierop te reageren?
De deelvragen die hierbij gesteld worden zijn: Wat is een Essentiële Tremor? Hoe worden tremoren gedetecteerd?
En hoe kan de data die uit een detectie komt worden gebruikt?

Om de vragen te beantwoorden, is dit document op de volgende manier opgebouwd. 
In hoofdstuk~\ref{section:tremor} wordt beschreven wat Essentiële Tremor inhoudt,
hoofdstuk~\ref{section:detection} beschrijft de huidige detectiemethoden voor Essentiële Tremor.
Hoofdstuk~\ref{section:prototyping} bekijkt een prototype dat is gemaakt voor de detectie van tremoren,
waarna hoofdstuk~\ref{section:application} mogelijke toepassingen beschrijft.
Als laatste zullen hoofdstukken~\ref{section:conclusion} en \ref{section:discussion} een conclusie en reflectie bevatten.